\documentclass[a4paper,amsmath]{oblivoir}

\usepackage{fapapersize}
\usefapapersize{*,*,1in,*,1in,*}

\makeatletter
\let\ATonum\@onum
\makeatother

\usepackage{xcolor}
\usepackage{graphicx}
\usepackage{hologo}
\usepackage{relsize}
\usepackage{tcolorbox}
\tcbuselibrary{listings,breakable}
\tcbset{listing engine=listings,colframe=cyan,colback=cyan!5!white}

\setmainfont{Noto Serif}
\setsansfont{Noto Sans}
\setmonofont{Noto Sans Mono}
\setkomainfont(Noto Serif CJK KR)(* Bold)(Noto Sans CJK KR)
\setkosansfont[Noto Sans CJK KR]()( Bold)( Medium)
\usepackage{unicode-math}
\setmathfont{LibertinusMath-Regular.otf}

\newcounter{sub}
\newcommand\bangotsuite{\stepcounter{sub}\thesub}

\setlength{\parindent}{0mm}

\newcommand\pkg[1]{\textsf{#1}}

\ExplSyntaxOn 

\NewDocumentEnvironment {intro} {o}
  {
    \IfValueTF { #1 }
      {
        \int_set:Nn \l_tmpa_int { #1 }
      }
      {
        \int_set:Nn \l_tmpa_int { 1 }
      }
    \noindent \rule {\linewidth}{3pt}
    \par 
    \sffamily [No.\space\int_use:N \l_tmpa_int ]\ \ 
    \bfseries
  }
  {
    \hfill 2019년~07월~10일
    \par 
    \vskip -.3\baselineskip 
    \noindent \rule {\linewidth }{1pt}
    \par 
    \vskip .5\baselineskip 
  }

\NewDocumentCommand \exverb { d|| }
  {
    \texttt { \tl_to_str:n { #1 } }
  }

\ExplSyntaxOff 

\NewDocumentEnvironment {questiona} { m }
  {
    \begin{tcolorbox}[title={#1},fonttitle={\sffamily\bfseries}]
      }{%
    \end{tcolorbox}
  }

\NewDocumentEnvironment {questionp} { }
  {
    \begin{tcolorbox}[colframe=orange!30!black!60,colbacktitle=orange!25!gray,
      title={연습문제},fonttitle={\sffamily\bfseries}]
      }{%
    \end{tcolorbox}
  }


\NewDocumentEnvironment {exampleonly} {}
{%
  \expandafter\tcblisting{listing only,breakable,before={\par\medskip\setstretch{1}}}
  }{
  \endtcblisting
}

\NewDocumentEnvironment {exampleside} {}
{%
  \tcblisting{listing side text,breakable,before={\par\medskip\setstretch{1}}}%
  }{%
  \endtcblisting
}

\newtcblisting{examplebelow}{breakable,before={\par\medskip\setstretch{1}}}

\begin{document}

\begin{intro}[3]
  Merry-Go-Round
\end{intro}

이재호 풀이.
\medskip

\begin{questionp}
  \fbox{응용} 1. 다음과 같이 이루어진 수열이 있다. 인자로 주어지는 $n$번째 항까지의 합을 출력하여라.
  \begin{quote}
    3, 7, 15, 1, 292, 1, 1, 1, 2, 1, 3, 1, 14, 2, 1, 1, 2, 2, 2, 2, 1, 84, 2, 1, 1, 15, 3, 13, 1, 4, 2, 6, 6, 99, 1, 2, 2, 6, 3, 5, 1, 1, 6, 8, 1, 7, 1, 2, 3, 7, 1, 2, 1, 1, 12, 1, 1, 1, 3, 1, 1, 8, 1, 1, 2, 1, 6, 1, 1, 5, 2, 2, 3, 1, 2, 4, 4, 16, 1, 161, 45, 1, 22, 1, 2, 2, 1, 4, 1, 2, 24, 1, 2, 1, 3, 1, 2
  \end{quote}

  \tcblower

  \begin{minipage}[t]{10cm}
    입력: \verb|\testsum{5}|\\
    출력: 318
  \end{minipage}
\end{questionp}

먼저 \verb|clist_map_inline|을 사용해서 입력 수를 넘으면
\verb|clist_map_break:|를 사용하여 탈출하고, 그 전에는 \verb|clist|의 원소를
정수 표현으로 더하여 계산하는 방식입니다.
왠지 \verb|clist_item|을 사용하여 \verb|step| 함수를 사용하면 비효율적일 것
같아 해당 방법은 아래에 \verb|clist_pop|과 함께 사용했습니다.
\begin{examplebelow}
\ExplSyntaxOn
\clist_clear_new:N \g_testsuminput_clist
\clist_gset:Nn \g_testsuminput_clist
  {
    3, 7, 15, 1, 292, 1, 1, 1, 2, 1, 3, 1, 14, 2, 1, 1, 2, 2, 2, 2, 1, 84, 2, 1, 1, 15, 3, 13, 1, 4, 2, 6, 6, 99, 1, 2, 2, 6, 3, 5, 1, 1, 6, 8, 1, 7, 1, 2, 3, 7, 1, 2, 1, 1, 12, 1, 1, 1, 3, 1, 1, 8, 1, 1, 2, 1, 6, 1, 1, 5, 2, 2, 3, 1, 2, 4, 4, 16, 1, 161, 45, 1, 22, 1, 2, 2, 1, 4, 1, 2, 24, 1, 2, 1, 3, 1, 2
  }

\int_new:N \g_testsuminputcnt_int
\int_gset:Nn \g_testsuminputcnt_int { \clist_count:N \g_testsuminput_clist }

\NewDocumentCommand \testsum { m }
  {
    \int_zero:N \l_tmpa_int
    \int_zero:N \l_tmpb_int
    \clist_map_inline:Nn \g_testsuminput_clist
      {
        \int_incr:N \l_tmpa_int
        \int_compare:nTF { \l_tmpa_int <= #1 }
          {
            \int_add:Nn \l_tmpb_int { ##1 }
          }
          { \clist_map_break: }
      }
    \int_use:N \l_tmpb_int
  }

\testsum{5}\\
\testsum{\g_testsuminputcnt_int}
\ExplSyntaxOff
\end{examplebelow}

\verb|clist_pop|과 함께 이번에 새로 배운 \verb|int_step_inline|을 활용한
방식입니다.

\begin{examplebelow}
\ExplSyntaxOn
\NewDocumentCommand \testsum { m }
  {
    \clist_set_eq:NN \l_tmpa_clist \g_testsuminput_clist

    \int_zero:N \l_tmpa_int
    \int_step_inline:nn { #1 }
      {
        \clist_pop:NN \l_tmpa_clist \l_tmpa_tl
        \int_add:Nn \l_tmpa_int { \l_tmpa_tl }
      }
    \int_use:N \l_tmpa_int
  }

\testsum{5}\\
\testsum{\g_testsuminputcnt_int}
\ExplSyntaxOff
\end{examplebelow}

\pagebreak
\begin{questionp}
  \fbox{기본} 2. 주어진 수까지 더하는 대신 곱한다면 계승(factorial)을 구하는 trivial한 함수를 만들 수 있다. 이를 작성하여라. 단 인자는 12이하의 정수로 한다. 

  \tcblower

  \begin{minipage}[t]{10cm}
    입력: \verb|\simplefact{10}| \\
    출력: 3628800
  \end{minipage}
\end{questionp}

전역 변수와 별도의 cs를 만들지 않고 싶어서 \verb|\l_tmpa_int| 스크래치
변수에 바로 곱셈을 행하였습니다.
\verb|\int_set:No|와 같은 함수가 없는 것 같아 \verb|exp_args:NNo|를
사용했습니다.
\begin{examplebelow}
\ExplSyntaxOn
\NewDocumentCommand \simplefact { m }
  {
    \int_set:Nn \l_tmpa_int { 1 }
    \int_step_inline:nn { #1 }
      {
        \exp_args:NNo \int_set:Nn \l_tmpa_int { \l_tmpa_int * ##1 }
      }
    \int_use:N \l_tmpa_int
  }
\ExplSyntaxOff

\simplefact{10}\\
\simplefact{12}
\end{examplebelow}

\pagebreak
\begin{questionp}
  \fbox{기본} 3. 30이하의 정수 인자를 받아서 $2^n$ 연산의 결과를 출력하는 함수 \verb|\bin_power:n|을 작성하여라. 단 \textsf{fp} 자료형의 power 연산자 \verb|**|를 쓰지 말고 \textsf{int}로 계산하여야 하며, 그 결과는 \verb|\fp_eval:n { 2**#1 }|과 같아야 한다.

  \tcblower

  3. \begin{minipage}[t]{10cm}
    입력: \verb|\bin_power:n {8}| \\
    출력: 256
  \end{minipage}
\end{questionp}

연습문제 기본 2와 동일한 방법을 사용했습니다.
\begin{examplebelow}
\ExplSyntaxOn
\cs_new:Npn \bin_power:n #1
  {
    \int_set:Nn \l_tmpa_int { 1 }
    \int_step_inline:nn { #1 }
      {
        \exp_args:NNo \int_set:Nn \l_tmpa_int { \l_tmpa_int * 2 }
      }
    \int_use:N \l_tmpa_int
  }

\bin_power:n {8}\\
\bin_power:n {10}
\ExplSyntaxOff
\end{examplebelow}

\pagebreak
\begin{questionp}
  \fbox{실력} 4. 인자로 2진수가 주어진다. 이를 10진수로 바꾸어서 출력하는 함수를, expl3가 제공하는 진법 변환 함수를 차용하지 않고 작성하여라. 필요하다면 연습문제 3에서 작성한 \verb|\bin_power:n|을 활용하여라.

  \tcblower

  4. \begin{minipage}[t]{10cm}
    입력: \verb|\mytodec{101110}| \\
    출력: 46
  \end{minipage}
\end{questionp}

들어온 값을 tl로 보고 문자열 조작을 한 방법입니다.
\begin{examplebelow}
\ExplSyntaxOn
\NewDocumentCommand \mytodec { m }
  {
    \tl_set:Nn \l_tmpa_tl { #1 }
    \tl_reverse:N \l_tmpa_tl

    \int_zero:N \l_tmpa_int
    \int_set:Nn \l_tmpb_int { 1 }
    \tl_map_inline:Nn \l_tmpa_tl
      {
        \int_add:Nn \l_tmpa_int { \l_tmpb_int * ##1 }
        \exp_args:NNo \int_set:Nn \l_tmpb_int { \l_tmpb_int * 2 }
      }
    \int_use:N \l_tmpa_int
  }
\ExplSyntaxOff

\mytodec{101110}\\
\mytodec{1001101000001010111110110}
\end{examplebelow}

\pagebreak
\begin{questionp}
  \fbox{기본} 1. 주어지는 자연수를 우리말로 읽어서 그 결과를 한글로 출력하여라. 단 입력되는 숫자는 5자리 이하로 한다.

  \tcblower

  1. \\
  입력: \verb|\numtohangul{2019}| \\
  출력: 이천십구
\end{questionp}

한글 맞춤법 제44항에 따라 만 단위 뒤에는 띄어 쓰도록 하였습니다.
\begin{examplebelow}
\ExplSyntaxOn
\tl_gset:Nn \g_prefix_tl { 일이삼사오육칠팔구 }
\tl_gset:Nn \g_suffix_tl { 십백천만 }

\NewDocumentCommand \numtohangul { m }
  {
    \int_set:Nn \l_tmpa_int { \tl_count:n { #1 } - 1 }
    \tl_map_inline:nn { #1 }
      {
        \bool_if:nT
          {
            \int_compare_p:n { \l_tmpa_int > 3 } ||
            \int_compare_p:n { \l_tmpa_int == 0 } ||
            \int_compare_p:n { ##1 != 1 }
          }
          {
            \tl_item:Nn \g_prefix_tl { ##1 }
          }
        \int_compare:nT { ##1 != 0 }
          {
            \tl_item:Nn \g_suffix_tl { \l_tmpa_int }
          }
        \int_compare:nT { \l_tmpa_int == 4 } { ~ }
        \int_decr:N \l_tmpa_int
      }
  }
\ExplSyntaxOff

\numtohangul{2019}\\
\numtohangul{11019}\\
\numtohangul{909}
\end{examplebelow}

\pagebreak
\begin{questionp}
  \fbox{실력} 2. 두 수를 인자로 받아 그 최대공약수를 출력하는 명령 \verb|\mygcd|를 작성하여라.

  \tcblower

  2. \\
  입력: \verb|\mygcd{16}{24}| \\
  출력: 8
\end{questionp}

Python 코드를 직역하였습니다.
\begin{examplebelow}
\ExplSyntaxOn
\int_new:N \l_tmpc_int
\NewDocumentCommand \mygcd { m m }
  {
    \int_compare:nTF { #1 > #2 }
      {
        \int_set:Nn \l_tmpa_int { #1 }
        \int_set:Nn \l_tmpb_int { #2 }
      }
      {
        \int_set:Nn \l_tmpa_int { #2 }
        \int_set:Nn \l_tmpb_int { #1 }
      }

      \int_while_do:nn { \l_tmpb_int != 0 }
        {
          \int_set:Nn \l_tmpc_int
            {
              \int_mod:nn { \l_tmpa_int } { \l_tmpb_int }
            }
          \int_set:Nn \l_tmpa_int { \l_tmpb_int }
          \int_set:Nn \l_tmpb_int { \l_tmpc_int }
        }

    \int_use:N \l_tmpa_int
  }
\ExplSyntaxOff

\mygcd{16}{24}\\
\mygcd{19077743}{15973807}
\end{examplebelow}

\end{document}

