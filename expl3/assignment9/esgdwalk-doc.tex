%
% `esgdwalk-doc.tex’
%    this document is part of esgdwalk package version 1.0 (2019/08/03)
% written by Nova de Hi.
% 
%
\documentclass[nokorean]{oblivoir}

%%% package version and date
\newcommand*\esgdwalkdate {2019/08/03}
\newcommand*\esgdwalkversion {1.0}

\usepackage{fapapersize}
\usefapapersize{210mm,297mm,1in,*,1in,*}

\ifLuaOrXeTeX
\usepackage{fontspec}
\setmainfont{TeX Gyre Termes}
\else
\usepackage{mathptmx}
\fi

\usepackage[color=orange]{esgdwalk}

\begin{document}

\title{esgdwalk package}
\author{Nova de Hi}
\date{\esgdwalkdate\quad ver \esgdwalkversion}

\maketitle

\begin{abstract}
\texttt{esgdwalk} package provides only one command \verb|\DWalk|.
\end{abstract}

\tableofcontents

\section{Introduction}

I made this package in order to provide a practical example of
packing an EXPL3 package. 

\section{Usage}

You can load this package with a line in the preamble of your document.
\begin{verbatim}
\usepackage[<option>]{esgdwalk}
\end{verbatim}

The \texttt{<option>} can be specified as following:
\begin{verbatim}
\usepackage[color=blue]{esgdwalk}
\end{verbatim}
then all the lines drawn with \verb|\DWalk| command will be set in blue color.

The only command provided by this package is \verb|\DWalk|.

\begin{verbatim}
\DWalk{angle=<angle>,forward=<forward>,walk=<walk>}
\end{verbatim}

This command makes a drawing area utilizing TikZ and draws (colored) lines according to
the rules of:\\
1. go forward in the length of \texttt{forward}\\
2. turn left in \texttt{angle} angle, \\
3. repeat these actions in \texttt{walk} times.

If \texttt{angle} is \verb|random|, then \verb|\DWalk| will
draw all lines with randomized value of angle.

If the value of the key \texttt{forward} is \verb|random|, the same as \verb|angle=random|.

When \texttt{forward} is given as \verb|forward=a+b|, then the length of the line
will be grow increasingly from \verb|a| by \verb|b| on each turn.

This package requires the \verb|tikz| and \verb|esgutil| packages. The latter
can be downloaded from my personal website, the url address of which is not known.

\section{Example}

\begin{verbatim}
\usepackage[color=orange]{esgdwalk}  %%% preamble
%...
\DWalk{angle=89,forward=0.02+0.02,walk=200}
\end{verbatim}

\DWalk{angle=89,forward=0.02+0.02,walk=200}

\section{Acknowledgements}

Many thanks go to the members of ESG.

\end{document}
