\documentclass[a4paper,amsmath,itemph]{oblivoir}

\usepackage{fapapersize}
\usefapapersize{*,*,1in,*,1in,*}

\makeatletter
\let\ATonum\@onum
\makeatother

\usepackage{relsize}
\usepackage{tcolorbox}
\tcbuselibrary{listings,breakable}
\tcbset{listing engine=listings,colframe=cyan,colback=cyan!5!white}

\setmainfont{Noto Serif}
\setsansfont{Noto Sans}
\setmonofont{Noto Sans Mono}
\setkomainfont[Noto Serif CJK KR]()( Bold)( Medium)
\setkosansfont[Noto Sans CJK KR]()( Bold)( Medium)
\usepackage{unicode-math}
\setmathfont{libertinusmath-regular.otf}

\setlength{\parindent}{0mm}

\newcommand\pkg[1]{\textsf{#1}}

\ExplSyntaxOn 

\NewDocumentEnvironment {intro} {o}
{
    \IfValueTF { #1 }
    {
        \int_set:Nn \l_tmpa_int { #1 }
    }
    {
        \int_set:Nn \l_tmpa_int { 1 }
    }
    \noindent \rule {\linewidth}{3pt}
    \par 
    \sffamily [No.\space\int_use:N \l_tmpa_int ]\ \ 
    \bfseries
}
{
    \hfill 년 \ \ \ 월 \ \ \ 일 
    \par 
    \vskip -.3\baselineskip 
    \noindent \rule {\linewidth }{1pt}
    \par 
    \vskip .5\baselineskip 
}

%%%%%%%%%%%%%% codes

\ExplSyntaxOff 


\begin{document}

\begin{intro}
Hello, world
\end{intro}

\begin{tcolorbox}[title={문제},fonttitle={\sffamily\bfseries}]
인자를 하나 받아서 ``Hello, \verb|#1|!’’ 형식으로 출력하는 명령 \verb|\hellocmd|를 작성하여라.
\tcblower 
입력: \verb|\hellocmd{world}|\\
출력: Hello, world!
\end{tcolorbox}


\paragraph{\pkg{xparse}}
\pkg{xparse}에서 다음 명령은 이미 알고 있다고 가정한다. 만약 다음 명령이 무엇을 하려는 것인지 잘 모르겠다면 xparse 문서를 반드시 숙지하여야 한다.
\begin{itemize}\firmlist
\item \verb|\NewDocumentCommand|.
\item \verb|\RenewDocumentCommand|.
\item \verb|\IfNoValueTF|.
\item \verb|\IfBooleanTF|.
\item 인자형 지시자: \textsf{m, o, O, s}.
\end{itemize}

\paragraph{interface3.pdf}

이 문서는 expl3의 (거의 유일하고 완전한) reference이다. 항상 열어두고 작업하는 것이 옳다. \verb|texdoc interface3|.


\bigskip

이제 주어진 문제를 풀어보자.

\medskip

\begin{tcblisting}{listing only}
\ExplSyntaxOn
\NewDocumentCommand \hellocmd { m }
{
    Hello, ~ #1 !
}
\ExplSyntaxOff
\end{tcblisting}

\verb|\ExplSyntaxOn|과 \verb|\ExplSyntaxOff|로 expl3 언어로 코딩하는 부분임을 표시한다. 이 범위 안에서는
\begin{enumerate}\firmlist
\item 스페이스는 모두 무시된다.
\item \verb|:|과 \verb|_|가 명령의 일부로 쓰인다(즉, letter이다).
\end{enumerate}
그러므로 행끝 문자를 없애기 위해서 \verb|%|를 행 끝에 분여야 했던 불편이 없다.
그 대신 스페이스를 입력 문자열에 남겨야 할 적에 반드시 \verb|~|(틸데)로 그것을 명시적으로 표시해주어야 한다.

위의 명령을 실행하면 다음처럼 된다.

\ExplSyntaxOn
\NewDocumentCommand \hellocmd { m }
{
    Hello, ~ #1 !
}
\ExplSyntaxOff

\medskip

\begin{tcblisting}{listing side text}
\hellocmd{world}
\end{tcblisting}

이번에는 다음과 같은 코드를 생각해보자.

\medskip

\begin{tcblisting}{listing only}
\ExplSyntaxOn
\cs_new:Npn \hello_fn:n #1
{
    Hello,~ #1 !
}

\NewDocumentCommand \hellocmd { m }
{
    \hello_fn:n { #1 }
}
\ExplSyntaxOff
\end{tcblisting}

이 코드는 문서 명령이 어떤 것인지를 명백하게 보여준다. 문서 명령 \verb|\hellocmd|가 하는 일은 인자를 하나 받아서 expl3로 작성된 ``함수’’ \verb|\hello_fn:n|에 넘겨주는 일만을 한다. 실제로 어떤 작용을 하는 것은 이 함수가 하게 되는 것이다.
그러나 \verb|\begin{document}| 이후에 \verb|\hello_fn:n|이라는 명령을 쓸 수는 없기 때문에 문서 명령 \verb|\hellocmd|를 정의하게 되었다.

한편, 우리가 넘겨받은 인자는 평범한 알파벳으로 이루어진 문자열(토큰열)이다. 앞으로 공부하게 될 중요한 자료형은 다음과 같은 것이 있다.

\begin{description}\firmlist
\item [token list] 가장 기본적인 문자열이다. 매크로나 그룹도 하나의 토큰으로 간주한다. 토큰열에서 모든 문자는 그 카테고리 코드를 보존하고 있다.
\item [string] 토큰열과 같지만 카테고리 코드가 모두 12(other)로 간주된다. 따라서 매크로는 그냥 문자열일 뿐이고 실행(확장)되지 않는다.
\item [integer] 정수형. 보통의 형검사를 하는 언어에서의 int와 매우 유사하다.
\item [floating point]. 부동소숫점 실수형. 유효숫자가 16자리인 실수이다.
\item [dimension] \TeX 의 길이값 형식으로서 단위(sp, pt, mm, etc.)를 포함한다.
\item [skip] \TeX 의 skip 형식으로서 둘 이상의 dim으로 구성된다.
\item [keys] <\textit{key}> = <\textit{value}> 형식의 데이터.
\end{description}

여기에 다음과 같은 리스트 형식의 자료형이 존재한다.
\begin{description}\firmlist
\item [sequence] expl3 특유의 리스트 자료형이다. 여러 개의 아이템을 일관되게 다루기 위한 것으로 stack처럼 활용할 수 있다.
\item [clist] comma separated list. seq의 특별한 경우로서 각 아이템이 \verb|,|로 구분된 것이다. 간략화한 seq라고 생각하면 된다.
\item [property list] key=value 형식의 리스트이다. 
\end{description}

이밖에, file, regex, sys, quark, cs, box, coffin 등의 자료형이 더 있지만 천천히 배워나갈 것이다.

\bigskip

연습삼아서 인자로 받은 문자열을 tl 변수에 넣어서 활용하는 방법을 생각해보자.

\medskip

\begin{tcblisting}{listing only}
\cs_new:Npn \hello_fn:n #1
{
    \tl_set:Nn \l_tmpa_tl { #1 }
    \l_tmpa_tl
}
\end{tcblisting}

\verb|\cs_new:Npn|은 새로운 함수(cs)를 정의하라는 명령으로서 자주 쓰일 것이므로 이 문법을 잘 보아두어야 한다.
이 자체가 하나의 함수이다. N은 control sequence, p는 parameter, n은 ``중괄호로 둘러싸인 토큰열’’을 가리킨다.
이 인자형 지시자는 매우 중요한데, 지금 다 알아둘 필요는 없지만 다음 몇 가지는 숙지해두자.

\begin{description}\firmlist
\item [n] 가장 많이 쓰이는 인자형이다. 중괄호로 묶어서 전달되는 토큰열. \verb|{...}|가 하나의 n에 해당한다.
\item [N] 한 개의 매크로를 가리킨다. 예를 들면 \verb|\l_tmpa_tl| 자체가 하나의 N이다.
\item [o, x, f, e] 인자의 확장형을 가리키는데 expansion 문제는 나중에 천천히 다루게 된다.
\item [V] 한 개의 매크로를 가리키지만 그 매크로 자체를 전달하는 것이 아니라 그 값(value)을 넘겨받는다는 의미이다.
\end{description}
따라서 \verb|\tl_set:Nn|은 그 뒤에 차례로 한 개의 매크로와 한 개의 중괄호 범위가 올 것임을 예상할 수 있다. 이 함수의 의미는 n으로 주어지는 토큰열을 N이라는 매크로에 할당하라는 것이다. 당연히 N은 tl 자료형이어야 한다.

이 때 \verb|\l_tmpa_tl|은 어떤 데이티를 저장할 매크로인데, 이를 ``변수’’라고 한다. 변수는 어떤 자료형인지를 명시하는 것이 좋다. l은 이것이 지역변수임을 의미하고 tmpa는 고유한 이름이며 tl은 데이터 타입이다.

이름이 \verb|tmpa|인 것과 \verb|tmpb|인 것은 expl3 자체에 정의된 scratch 변수이다. 즉 이 변수명을 선언하지 않아도 바로 사용할 수 있다. 스크래치 변수는 \verb|l|과 \verb|g|에 대하여 두 개씩 있어서 모두 네 개가 각 데이터별로 정의되어 있다.
자신만의 변수를 만들려면 \verb|\tl_new:N|으로 변수를 선언해야 할 때가 있다.

위의 코드에서 둘째 줄 \verb|\l_tmpa_tl|은 그 자체로 그 위치에 이 변수의 내용을 식자한다. 이렇게 하는 것을 ``입력문자열에 남긴다’’고 표현한다. 그리고 그것이 이 함수의 반환값 비슷한 것이 된다는 점에 유의하라. expl3는 명시적인 반환값은 별도로 존재하지 않고(따라서 엄밀한 의미의 function이 아니다), 입력 문자열에 어떤 것을 남김으로써 반환값처럼 처리하게 한다. 요컨대, expl3에는 \verb|return| 문이 없다.


\paragraph{용어}

\begin{description}\firmlist
\item[문서 명령] Expl3로 작성하는 control sequence들은 \verb|_|(underscore)와 \verb|:|(colon)을 포함하므로 문서 중에 이를 사용할 수 없다. expl3를 이용하여 작성한 함수를 문서 중에 사용하기 위하여 만드는 매크로를 ``문서 명령(document commands)’’라고 부른다. 문서 명령을 작성하도록 도와주는 것이 \pkg{xparse}이며 expl3와 필수적으로 함께 쓰여야 한다.
\item[입력 문자열] \emph{input stream}. \TeX 에게 전해주는 토큰(열)을 가리키는 말로 쓴다.
\item[인자형 지시자] expl3 함수의 일부로서 그 함수가 흡수(absorb)할 인자의 형식을 미리 지정한다. 참고로, \pkg{xparse}의 인자형 지시자(m, o, 등)는 expl3의 인자형 지시자와 기능과 목적이 다 다르므로 혼동하면 안 된다.
\item[자료형] expl3의 data type.
\item[함수] 인자를 취하여 일정한 작용을 하도록 정의한 expl3의 메크로(데이터) 형식
\item[변수] 데이터를 저장하고 있는 매크로
\item[범위] \verb|\begingroup|에서 \verb|\endgroup|까지, 즉 \verb|{|에서 \verb|}|까지를 가리킨다. 지역(local) 변수는 이 범위 내에서만 유효하다. 범위와 상관없이 문서 전체에 유효한 변수를 전역적(global)이라고 한다.
\end{description}


\vfill

\begin{tcolorbox}[title={연습문제},fonttitle=\sffamily]
\fbox{기본} 1. 문제에서 제시한 \verb|\hellocmd|에서 인자로 주어진 이름을 이탤릭체로 식자하도록 해보아라.

\bigskip

\fbox{발전} 2. 문제에서 제시한 \verb|\hellocmd|를, 넘어온 인자의 첫 글자를 대문자로 바꾸어 출력하도록 작성하여라.
\tcblower
입력: \verb|\Hellocmd{world}|\\
출력: Hello, \textit{world}! 또는 Hello, World!
\end{tcolorbox}

\vfill

\hfill Nova de Hi.


\end{document}
