\documentclass[a4paper,amsmath]{oblivoir}

\usepackage{fapapersize}
\usefapapersize{*,*,1in,*,1in,*}

\makeatletter
\let\ATonum\@onum
\makeatother

\usepackage{esgutil}
\usepackage{refcount}


\usepackage{xcolor}
\usepackage{graphicx}
\usepackage{hologo}
\usepackage{relsize}
\usepackage{tcolorbox}
\tcbuselibrary{listings,breakable}
\tcbset{listing engine=listings,colframe=cyan,colback=cyan!5!white}

\setmainfont{TeX Gyre Pagella}
\setsansfont{Noto Sans}
\setmonofont{Noto Sans Mono}
\setkomainfont(Noto Serif CJK KR)(* Bold)(Noto Sans CJK KR)
\setkosansfont[Noto Sans CJK KR]()( Bold)( Medium)
\usepackage{unicode-math}
\setmathfont{texgyrepagella-math.otf}

\newcounter{sub}
\newcommand\bangotsuite{\stepcounter{sub}\thesub}

\usepackage{tikz}
\newcommand\tikzlogo{Ti\textit{k}Z}

\setlength{\parindent}{0mm}

\newcommand\pkg[1]{\textsf{#1}}

\ExplSyntaxOn 

\NewDocumentEnvironment {intro} {o}
{
  \IfValueTF { #1 }
  {
    \int_set:Nn \l_tmpa_int { #1 }
  }
  {
    \int_set:Nn \l_tmpa_int { 1 }
  }
  \noindent \rule {\linewidth}{3pt}
  \par 
  \sffamily [No.\space\int_use:N \l_tmpa_int ]\ \ 
  \bfseries
}
{
  \hfill 2019년 7월 21
  \par 
  \vskip -.3\baselineskip 
  \noindent \rule {\linewidth }{1pt}
  \par 
  \vskip .5\baselineskip 
}

\NewDocumentCommand \exverb { d|| }
{
  \texttt { \tl_to_str:n { #1 } }
}

\ExplSyntaxOff 

\NewDocumentEnvironment {questiona} { m }
{
  %    \medskip
  \begin{tcolorbox}[title={#1},fonttitle={\sffamily\bfseries}]
    }{%
  \end{tcolorbox}
}

\NewDocumentEnvironment {questionp} { }
{
  %    \medskip
  \begin{tcolorbox}[colframe=orange!30!black!60,colbacktitle=orange!25!gray,
    title={연습문제},fonttitle={\sffamily\bfseries}]
    }{%
  \end{tcolorbox}
}


\NewDocumentEnvironment {exampleonly} {}
{%
  %    \begin{tcblisting}{listing only}
  \expandafter\tcblisting{listing only,breakable,before={\par\medskip\setstretch{1}}}
  }{
  \endtcblisting
  %    \end{tcblisting}
}

\NewDocumentEnvironment {exampleside} {}
{%
  \tcblisting{listing side text,righthand width=.35\textwidth,breakable,before={\par\medskip\setstretch{1}}}%
  }{%
  \endtcblisting
}

%\NewDocumentEnvironment {examplebelow} {}
%{%
%    \medskip
%    \tcblisting{text below listing,breakable}%
%}{%
%    \endtcblisting
%}
\newtcblisting{examplebelow}{breakable,before={\par\medskip\setstretch{1}}}

\begin{document}

\begin{intro}[5]
  Dim and Dimmer
\end{intro}

\fbox{기본} 1. 다음 도표의 빈 칸을 채워 완성하여라.

\begin{examplebelow}
\ExplSyntaxOn

\cs_new:Npn \convert_from:n #1
  {
    \dim_to_decimal_in_unit:nn { #1 } { 1pt } &
    \dim_to_decimal_in_unit:nn { #1 } { 1mm } &
    \dim_to_decimal_in_unit:nn { #1 } { 1pc } &
    \dim_to_decimal_in_unit:nn { #1 } { 1in } \\ \hline
  }

\begin{center}
\begin{tabular}{c|c|c|c|c}
     & pt & mm & pc & in \\ \hline
  pt & \convert_from:n { 1pt }
  mm & \convert_from:n { 1mm }
  pc & \convert_from:n { 1pc }
  in & \convert_from:n { 1in }
\end{tabular}
\end{center}

\ExplSyntaxOff
\end{examplebelow}

\newpage
\fbox{기본} 2. \LaTeX 에 \verb|\settowidth|와 \verb|\settoheight|, \verb|\settodepth|라는 명령이 있다. 이의 사용법은 다음과 같다.

\begin{exampleside}
  \newlength{\mylen}
  \settowidth{\mylen}{beautiful}
  \the\mylen
\end{exampleside}

이것과 동일한 작용을 하는 \verb|\SettoWidth|, \verb|\SettoHeight|, \verb|\SettoDepth|를 expl3로 만들고 \verb|\SettoTotalheight|도 작성해보아라. 

\begin{examplebelow}
\ExplSyntaxOn
\NewDocumentCommand \SettoWidth { m m }
  {
    \dim_if_exist:NTF { #1 }
      {
        \hbox_set:Nn \l_tmpa_box { #2 }
        \dim_set:Nn \l_tmpa_dim { \box_wd:N \l_tmpa_box }
        \dim_set_eq:NN #1 \l_tmpa_dim
      }
      {
        \texttt{\bs #1}~is~invalid
      }
  }

\NewDocumentCommand \SettoHeight { m m }
  {
    \dim_if_exist:NTF { #1 }
      {
        \hbox_set:Nn \l_tmpa_box { #2 }
        \dim_set:Nn \l_tmpa_dim { \box_ht:N \l_tmpa_box }
        \dim_set_eq:NN #1 \l_tmpa_dim
      }
      {
        \texttt{\bs #1}~is~invalid
      }
  }

\NewDocumentCommand \SettoDepth { m m }
  {
    \dim_if_exist:NTF { #1 }
      {
        \hbox_set:Nn \l_tmpa_box { #2 }
        \dim_set:Nn \l_tmpa_dim { \box_dp:N \l_tmpa_box }
        \dim_set_eq:NN #1 \l_tmpa_dim
      }
      {
        \texttt{\bs #1}~is~invalid
      }
  }

\NewDocumentCommand \SettoTotalheight { m m }
  {
    \dim_if_exist:NTF { #1 }
      {
        \hbox_set:Nn \l_tmpa_box { #2 }
        \dim_set:Nn \l_tmpa_dim { \box_ht:N \l_tmpa_box }
        \dim_set:Nn \l_tmpb_dim { \box_dp:N \l_tmpa_box }
        \dim_set:Nn #1 { \l_tmpa_dim + \l_tmpb_dim }
      }
      {
        \texttt{\bs #1}~is~invalid
      }
  }

\dim_new:N \mywidth
\SettoWidth{\mywidth}{beautiful}
\verb|\mywidth|~is~\dim_use:N \mywidth\\

\dim_new:N \myheight
\SettoHeight{\myheight}{beautiful}
\verb|\myheight|~is~\dim_use:N \myheight\\

\dim_new:N \mydepth
\SettoDepth{\mydepth}{beautiful}
\verb|\mydepth|~is~\dim_use:N \mydepth\\

\dim_new:N \mytotalheight
\SettoTotalheight{\mytotalheight}{beautiful}
\verb|\mytotalheight|~is~\dim_use:N \mytotalheight
\ExplSyntaxOff
\end{examplebelow}


\newpage
\fbox{기본} 3. 다음 문장에서, 각 단어의 (문장부호를 제외한) 길이를 측정하여 평균값을 구하고, 그 문단의 margin에 \verb|\rule{<평균값>}{10pt}|의 막대를 그려라.

\begin{framed}
  남방의 비는 차갑고 단단하고 찬란한 눈꽃으로 변하는 일이 절대 없다. 식자들은 남방의 비가 단조롭다고 생각하는데 비 자신도 그것을 불행으로 여기는지 어떤지, 강남의 눈은 그래도 촉촉하고 윤기가 돌아 아름답기 그지없다. 그것은 아직 눈뜨지 않은 청춘의 소식이며, 건강한 처녀의 살결이다. 

  눈 덮인 들에는 핏빛의 붉은 동백이며, 흰빛에 푸르름이 감도는 외겹 매화며, 경쇠 모양의 진노랑 새양꽃이 있고, 눈 밑에는 아직도 파랗게 언 잡초들이 있다. 나비는 분명 없었다. 꿀벌이 동백꽃과 매화꽃의 꿀을 따러 왔었는지 아닌지, 나는 확실히 기억할 수 없다. 단지 나의 눈앞에는 겨울꽃이 핀 눈 덮인 들판에 바쁘게 날아다니는 수많은 벌들이 보이는 듯, 시끄럽게 붕붕거리는 소리가 들리는 듯했다.
\end{framed}

\begin{exampleonly}
\ExplSyntaxOn
\NewDocumentCommand \SettoWidth { m m }
  {
    \dim_if_exist:NTF { #1 }
      {
        \hbox_set:Nn \l_tmpa_box { #2 }
        \dim_set:Nn \l_tmpa_dim { \box_wd:N \l_tmpa_box }
        \dim_set_eq:NN #1 \l_tmpa_dim
      }
      {
        \texttt{\bs #1}~is~invalid
      }
  }

\dim_new:N \l_word_length_dim
\cs_new:Npn \tl_average_word_length:n #1
  {
    \tl_if_blank:nF { #1 }
      {
        \seq_set_split:Nnn \l_tmpa_seq { ~ } { #1 }
        \dim_zero:N \l_word_length_dim
        \seq_map_inline:Nn \l_tmpa_seq
          {
            \SettoWidth { \l_tmpb_dim } { ##1 }
            \dim_add:Nn \l_word_length_dim { \l_tmpb_dim }
          }
        #1.~
        \marginpar
          {
            \rule
              {
                \dim_eval:n { \l_word_length_dim / \seq_count:N \l_tmpa_seq }
              }
              {
                10pt
              }
          }
      }
  }

\NewDocumentCommand \averagewordlength { m }
  {
    \seq_set_split:Nnn \l_tmpa_seq { . } { #1 }
    \seq_map_function:NN \l_tmpa_seq \tl_average_word_length:n
  }
\ExplSyntaxOff

\averagewordlength
  {
    남방의 비는 차갑고 단단하고 찬란한 눈꽃으로 변하는 일이 절대 없다. 식자들은 남방의 비가 단조롭다고 생각하는데 비 자신도 그것을 불행으로 여기는지 어떤지, 강남의 눈은 그래도 촉촉하고 윤기가 돌아 아름답기 그지없다. 그것은 아직 눈뜨지 않은 청춘의 소식이며, 건강한 처녀의 살결이다.
    눈 덮인 들에는 핏빛의 붉은 동백이며, 흰빛에 푸르름이 감도는 외겹 매화며, 경쇠 모양의 진노랑 새양꽃이 있고, 눈 밑에는 아직도 파랗게 언 잡초들이 있다. 나비는 분명 없었다. 꿀벌이 동백꽃과 매화꽃의 꿀을 따러 왔었는지 아닌지, 나는 확실히 기억할 수 없다. 단지 나의 눈앞에는 겨울꽃이 핀 눈 덮인 들판에 바쁘게 날아다니는 수많은 벌들이 보이는 듯, 시끄럽게 붕붕거리는 소리가 들리는 듯했다.
  }
\end{exampleonly}

\ExplSyntaxOn
\NewDocumentCommand \SettoWidth { m m }
  {
    \dim_if_exist:NTF { #1 }
      {
        \hbox_set:Nn \l_tmpa_box { #2 }
        \dim_set:Nn \l_tmpa_dim { \box_wd:N \l_tmpa_box }
        \dim_set_eq:NN #1 \l_tmpa_dim
      }
      {
        \texttt{\bs #1}~is~invalid
      }
  }

\dim_new:N \l_word_length_dim
\cs_new:Npn \tl_average_word_length:n #1
  {
    \tl_if_blank:nF { #1 }
      {
        \seq_set_split:Nnn \l_tmpa_seq { ~ } { #1 }
        \dim_zero:N \l_word_length_dim
        \seq_map_inline:Nn \l_tmpa_seq
          {
            \SettoWidth { \l_tmpb_dim } { ##1 }
            \dim_add:Nn \l_word_length_dim { \l_tmpb_dim }
          }
        #1.~
        \marginpar
          {
            \rule
              {
                \dim_eval:n { \l_word_length_dim / \seq_count:N \l_tmpa_seq }
              }
              {
                10pt
              }
          }
      }
  }

\NewDocumentCommand \averagewordlength { m }
  {
    \seq_set_split:Nnn \l_tmpa_seq { . } { #1 }
    \seq_map_function:NN \l_tmpa_seq \tl_average_word_length:n
  }
\ExplSyntaxOff

\averagewordlength
  {
    남방의 비는 차갑고 단단하고 찬란한 눈꽃으로 변하는 일이 절대 없다. 식자들은 남방의 비가 단조롭다고 생각하는데 비 자신도 그것을 불행으로 여기는지 어떤지, 강남의 눈은 그래도 촉촉하고 윤기가 돌아 아름답기 그지없다. 그것은 아직 눈뜨지 않은 청춘의 소식이며, 건강한 처녀의 살결이다.
    눈 덮인 들에는 핏빛의 붉은 동백이며, 흰빛에 푸르름이 감도는 외겹 매화며, 경쇠 모양의 진노랑 새양꽃이 있고, 눈 밑에는 아직도 파랗게 언 잡초들이 있다. 나비는 분명 없었다. 꿀벌이 동백꽃과 매화꽃의 꿀을 따러 왔었는지 아닌지, 나는 확실히 기억할 수 없다. 단지 나의 눈앞에는 겨울꽃이 핀 눈 덮인 들판에 바쁘게 날아다니는 수많은 벌들이 보이는 듯, 시끄럽게 붕붕거리는 소리가 들리는 듯했다.
  }


\newpage
\fbox{기본} 4. 다음 표를 완성하고 1em의 크기가 fontsize에 따라 어떻게 달라지는지 조사하여라. 만약 문서 폰트 크기 옵션이 10pt가 아니라 11pt 또는 12pt가 되면 이 값이 어떻게 변하는지도 조사하여라.

\begin{examplebelow}
\begin{center}
  \ExplSyntaxOn

  \begin{tabular}{r|r}
    \hline
    font~size~command & value~of~em \\ \hline
    \bs tiny & 6.0pt,~7.0pt,~8.0pt \\ \hline
    \bs scriptsize & 7.0pt,~8.0pt,~9.0pt \\ \hline
    \bs footnotesize & 8.0pt,~9.0pt,~10.0pt \\ \hline
    \bs small & 9.0pt,~10.0pt,~10.95pt \\ \hline
    \bs normalsize & 10.0pt,~10.95pt,~12.0pt \\ \hline
    \bs large & 10.95pt,~12pt,~14.4pt \\ \hline
    \bs Large & 12.0pt,~14.4pt,~17.28pt \\ \hline
    \bs LARGE & 14.4pt,~17.28pt,~20.74pt \\ \hline
    \bs huge & 17.28pt,~20.74pt,~24.88pt \\ \hline
    \bs Huge & 20.74pt,~24.88pt,~24.88pt \\ \hline
  \end{tabular}\

  \ExplSyntaxOff
\end{center}
\end{examplebelow}


\newpage
\makeatletter
\newlength\LC@temph\newlength\LC@tempw\newlength\LC@tempwtwo
\newlength\LC@templ
\newcommand*{\lccompthreevar}[5]{%
  \settowidth{\LC@temph}{$\scriptstyle #2$}
  \settowidth{\LC@tempw}{$\scriptstyle #3$}
  \settowidth{\LC@tempwtwo}{$\scriptstyle #3$}
  \settowidth{\LC@templ}{$\scriptstyle #4 #5$}
  {#1}{}_{\substack{#2 #3 #4 #5\\\makebox[\LC@temph]{\hfill$\scriptstyle 1$\hfill}%
      \makebox[\LC@tempw]{\hfill\phantom{$\scriptstyle 2$}\hfill}%
      \hspace{\LC@templ}%
  }}^{\hspace{\LC@temph}\hspace{\LC@tempw}\makebox[\LC@tempwtwo]{\hfill$\scriptstyle 3$\hfill}}
}
\makeatother

\fbox{실력} 5. 다음과 같은 식을 만들어야 한다고 하자:
${}_n\lccompthreevar{Q}{x}{y}{z}{}$

\medskip
이와 같은 식을 식자하는 \verb|\mycmd{Q}{n}{x}{y}{z}{1}{3}|와 같은 명령을 작성하되,
각 인자의 의미가 다음과 같도록 하여라.
\begin{enumerate}[\expandafter\ATonum1] \tightlist
  \item \verb|#2|는 \verb|#1|의 왼쪽에 오는 하첨자이다.
  \item \verb|#3, #4, #5| 세 개는 \verb|#1|의 오른쪽에 오는 하첨자이다.
  \item \verb|#6|에는 한 자리 숫자가 오는데 그 값은 3이하이다. 이 숫자에 따라 x, y, z의 해당 위치 아래(예컨대 1이면 x의 아래, 2이면 y의 아래)에 scriptsize로 식자한다.
  \item \verb|#7|은 \verb|#6|과 같이 하되, 하첨자의 위쪽에 식자한다.
\end{enumerate}

\begin{examplebelow}
\ExplSyntaxOn
\dim_new:N \l_temph_dim
\dim_new:N \l_tempw_dim
\dim_new:N \l_tempwtwo_dim
\dim_new:N \l_templ_dim

\NewDocumentCommand \MyCmd { m m m m m m m }
  {
    \SettoWidth{\l_temph_dim}{$\scriptstyle #3$}
    \SettoWidth{\l_tempw_dim}{$\scriptstyle #4$}
    \SettoWidth{\l_templ_dim}{$\scriptstyle #5$}
    { }\c_math_subscript_token{ #2 }
    #1
    { }\c_math_subscript_token
      {
        \substack
          {
            #3 #4 #5 {}\\
            \hbox_to_wd:nn { \l_temph_dim } { $\scriptstyle #6$ }
            \hbox_to_wd:nn { \l_tempw_dim + \l_templ_dim } { }
          }
      }
    \c_math_superscript_token
      {
        \hbox_to_wd:nn { \l_temph_dim + \l_tempw_dim } { }
        \hbox_to_wd:nn { \l_templ_dim } { $\scriptstyle #7$ }
      }
  }
\ExplSyntaxOff

\[
  \MyCmd{Q}{n}{x}{y}{z}{1}{3}
\]
\end{examplebelow}


\newpage
\fbox{기본} 1. 심심해진 철수는 idle 프로그램으로 다음과 같은 것을 하면서 놀았다.
이것을 본 영희가 “나는 expl3로 할 수 있어”라고 하였다. 과연 할 수 있었을까?

\begin{examplebelow}
\ExplSyntaxOn
\cs_new:Npn \approx_pi:n #1
  {
    \int_zero:N \l_tmpa_int
    \int_step_inline:nn { #1 }
      {
        \fp_set:Nn \l_tmpa_fp { rand() }
        \fp_set:Nn \l_tmpb_fp { rand() }
        \fp_compare:nT
          {
            \l_tmpa_fp**2 + \l_tmpb_fp**2 <= 1
          }
          {
            \int_incr:N \l_tmpa_int
          }
      }
    \fp_eval:n { \l_tmpa_int / #1 * 4 }
  }

\approx_pi:n { 10000 }
\ExplSyntaxOff
\end{examplebelow}


\newpage
\fbox{기본} 2. 등거리 random walk을 모사하려고 한다. 회전각을 난수적으로 얻어서
\begin{examplebelow}
\ExplSyntaxOn
\cs_new:Npn \generate_random_walk:n #1
  {
    \tl_set:Nn \l_tmpa_tl { (0, 0) -- (0, 0.5) }
    \int_step_inline:nn { #1 }
      {
      \tl_put_right:Nx \l_tmpa_tl
        {
          -- ([turn] \int_rand:n { 360 } \c_colon_str 0.5)
        }
      }
    \draw \l_tmpa_tl ;
  }

\NewDocumentCommand \generaterandomwalk { m }
  {
    \begin{center}
    \begin{tikzpicture}
      \generate_random_walk:n { #1 }
    \end{tikzpicture}
    \end{center}
  }
\ExplSyntaxOff

\generaterandomwalk{100}
\end{examplebelow}

\end{document}

