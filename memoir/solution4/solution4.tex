\documentclass[b5paper]{memoir}

\usepackage{tikz}
\usepackage{xparse}
\usepackage{xcolor}
\usepackage{graphicx}

%%% Global Fonts Settings
\usepackage{fontspec}
\setmainfont{XITS}
\setsansfont{TeX Gyre Heros}
\usepackage{amsmath,amssymb,amsthm}
\usepackage{unicode-math}
\setmathfont{XITS Math}
%
%% chapter name font : LinBiolinum_RB.otf
%% chapter title font: LinBiolinum_R.otf
%

%%%%%%%%%%%%%%%%%%%%%%%%%%
%% page layout
%% leftmargin : 20mm, 
%% rightmargin: 35mm
%% uppermargin: 33mm
%% lowermargin: 33mm
\setstocksize{711pt}{501pt}
\settrimmedsize{\stockheight}{\stockwidth}{*}
\settrims{0pt}{0pt}
\settypeblocksize{526pt}{344pt}{*}
\setlrmargins{57pt}{*}{*}
\setulmargins{94pt}{*}{*}
\setcolsepandrule{10pt}{0pt}
\setheadfoot{12pt}{24pt}
\setheaderspaces{*}{12pt}{*}
\ExplSyntaxOn
\dim_set:Nn \marginparwidth { \leftmargin - \rightmargin - \marginparsep }
\ExplSyntaxOff
\setmarginnotes{7pt}{\marginparwidth}{5pt}
\checkandfixthelayout
%%%%%%%%%%%%%%%%%%%%%%%%%%

\newtheorem{assump}{Assumption}
\newtheorem{axiom}{Axiom}
\let\markupwd\textbf
\def\chex{}
\def\printHintsAnswers{}
\def\printTerm{}

\begin{document}

%%% titlepage
\thispagestyle{titlingpage}
\begin{tikzpicture}[remember picture,overlay]
  \filldraw[fill=blue!30!black, draw=white] (current page.east) --
    ++(-15cm, 0cm) -- ++(0cm, -7cm) -- ++(15cm, 0cm) -- cycle;
  \node [xshift=-14.6cm,yshift=-0.4cm,anchor=north west,text=white]
    at (current page.east)
  {%
    \sffamily\HUGE%
    \begin{tabular}{l}
      ADVANCED\\MICROECONOMIC\\THEORY\\{\LARGE Jaeho Lee}
    \end{tabular}
  };
\end{tikzpicture}
\clearpage

\mainmatter

\chapterstyle{MEst}
\pagestyle{MEst}

\chapter{Consumer Theory}

In the first two chapters of this volume, we will explore the essential features of modern
consumer theory --- a bedstock foundation on which so many theoretical structures in
economics are built. Some time later in your study of economics, you will begin to notice
just how central this theory is to the economist’s way of thinking. Time and time again
you will hear the echoes of consumer theory in virtually every branch of the discipline 
--- how it is conceived, how it is constructed, and how it is applied.

\section{Primitive Notions}

There are four building blocks in any model of consumer choice. 
They are the consumption set, the feasible set, the preference relation, and the behavioural assumption. 
Each is conceptually distinct from the others, though it is quite common sometimes to lose sight
of that fact. This basic structure is extremely general, and so, very flexible. By
specifying the form each of these takes in a given problem, many different situations
involving choice can be formally described and analysed. Although we will tend to
concentrate here on specific formalisations that have come to dominate economists’ view of
an individual consumer’s behaviour, it is well to keep in mind that ‘consumer theory’
\emph{per se} is in fact a very rich and flexible \emph{theory of choice}.

The notion of a \markupwd{consumption set} is straightforward. We let the consumption set, $X$,
represent the set of all alternatives, or complete consumption plans, that the consumer
can conceive --- whether some of them will be achievable in practice or not. What we
intend to capture here is the universe of alternative choices over which the consumer’s
mind is capable of wandering, unfettered by consideration of the realities of his present
situation. The consumption set is sometimes also called the \markupwd{choice set}.

Let each commodity be measured in some infinitely divisible units. Let $x_i \in \mathbb{R}$
represent the number of units of good $i$. We assume that only non-negative units of each
good are meaningful and that it is always possible to conceive of having \emph{no} units
of any particular commodity. Further, we assume there is a finite, fixed, but arbitrary
number $n$ of different goods. We let $\mathbf x = (x_1, \ldots, x_n )$ be a vector
containing different quantities of each of the $n$ commodities and call $\mathbf x$ a
\markupwd{consumption bundle} or a \markupwd{consumption plan}. A consumption bundle
$\mathbf x \in X$ is thus represented by a point $\mathbf x \in \mathbb{R}^n_+$. Usually,
we’ll simplify things and just think of the consumption set as the \emph{entire}
non-negative orthant, $X = \mathbb{R}^n_+$. In this case, it is easy to see that each 
of the following basic requirements is satisfied.

\begin{assump}[Properties of the Consumption Set $X$]
The minimal requirements on the consumption set are
\begin{enumerate} 
\item $X \subseteq \mathbb{R}^n_+$.
\item $X$ is closed.
\item $X$ is convex.
\item $\mathbf 0 \in X$.
\end{enumerate}
\end{assump}

The notion of a \markupwd{feasible set} is likewise very straightforward. We let $B$ represent
all those alternative consumption plans that are both conceivable and, more important,
realistically obtainable given the consumer’s circumstances. What we intend to capture
here are precisely those alternatives that are \emph{achievable} given the economic
realities the consumer faces. The feasible set $B$ then is that subset of the consumption
set $X$ that remains after we have accounted for any constraints on the consumer’s access
to commodities due to the practical, institutional, or economic realities of the world.
How we specify those realities in a given situation will determine the precise
configuration and additional properties that $B$ must have. 
For now, we will simply say that $B \subset X$.

A \markupwd{preference relation} typically specifies the limits, if any, on the consumer’s
ability to perceive in situations involving choice the form of consistency or
inconsistency in the consumer’s choices, and information about the consumer’s tastes for
the different objects of choice. The preference relation plays a crucial role in any
theory of choice. Its special form in the theory of consumer behaviour is sufficiently
subtle to warrant special examination in the next section.

Finally, the model is ‘closed’ by specifying some \markupwd{behavioural assumption}. This
expresses the guiding principle the consumer uses to make final choices and so identifies
the ultimate objectives in choice. It is supposed that \emph{the consumer seeks to
identify and select an available alternative that is most preferred in the light of his
personal tastes}.


\section{Preferences and Utility}

In this section, we examine the consumer’s preference relation and explore its connection
to modern usage of the term ‘utility’. Before we begin, however, a brief word on the
evolution of economists’ thinking will help to place what follows in its proper context.

In earlier periods, the so-called ‘Law of Demand’ was built on some extremely strong
assumptions. In the classical theory of Edgeworth, Mill, and other proponents of the
utilitarian school of philosophy, ‘utility’ was thought to be something of substance.
‘Pleasure’ and ‘pain’ were held to be well-defined entities that could be measured and
compared between individuals. In addition, the ‘Principle of Diminishing Marginal Utility’
was accepted as a psychological ‘law’, and early statements of the Law of Demand depended
on it. These are awfully strong assumptions about the inner workings of human beings.

The more recent history of consumer theory has been marked by a drive to render its
foundations as general as possible. Economists have sought to pare away as many of the
traditional assumptions, explicit or implicit, as they could and still retain a coherent
theory with predictive power. Pareto (1896) can be credited with suspecting that the idea
of a measurable ‘utility’ was inessential to the theory of demand. Slutsky (1915)
undertook the first systematic examination of demand theory without the concept of a
measurable substance called utility. Hicks (1939) demonstrated that the Principle of
Diminishing Marginal Utility was neither necessary, nor sufficient, for the Law of Demand
to hold. Finally, Debreu (1959) completed the reduction of standard consumer theory to
those bare essentials we will consider here. Today’s theory bears close and important
relations to its earlier ancestors, but it is leaner, more precise, and more general.

\subsection{Preference Relations}

Consumer preferences are characterised axiomatically. In this method of modelling as few
meaningful and distinct assumptions as possible are set forth to characterise the
structure and properties of preferences. The rest of the theory then builds logically from
these axioms, and predictions of behaviour are developed through the process of deduction.

These \markupwd{axioms of consumer choice} are intended to give formal mathematical
expression to fundamental aspects of consumer behaviour and attitudes towards the objects
of choice. Together, they formalise the view that the consumer \emph{can} choose and that
choices are \emph{consistent} in a particular way.

Formally, we represent the consumer’s preferences by a \emph{binary relation}, $\succsim$,
defined on the consumption set, $X$. If $x_1 \succsim x_2$, we say that ‘$x_1$ is at least
as good as $x_2$’, for this consumer.

That we use a binary relation to characterise preferences is significant and worth a
moment’s reflection. It conveys the important point that, from the beginning, our theory
requires relatively little of the consumer it describes. We require only that consumers
make \emph{binary} comparisons, that is, that they only examine two consumption plans at a
time and make a decision regarding those two. The following axioms set forth basic
criteria with which those binary comparisons must conform.

\begin{axiom}[Completeness] \label{ax:1}
For all $x_1$ and $x_2$ in $X$, either $x_1 \succsim x_2$ or $x_2 \succsim x_1$.
\end{axiom}

Axiom~\ref{ax:1} formalises the notion that the consumer can make comparisons, that is,
that he has the ability to discriminate and the necessary knowledge to evaluate
alternatives. It says the consumer can examine any two distinct consumption plans $x_1$
and $x_2$ and decide whether $x_1$ is at least as good as $x_2$ or $x_2$ is at least as
good as $x_1$.

\begin{axiom}[Transitivity] \label{ax:2}
For any three elements $x_1$, $x_2$, and $x_3$ in $X$, if $x_1 \succsim x_2$ and $x_2
\succsim x_3$, then $x_1 \succsim x_3$.
\end{axiom}

Axiom~\ref{ax:2} gives a very particular form to the requirement that the consumer’s
choices be \emph{consistent}. Although we require only that the consumer be capable of
comparing two alternatives at a time, the assumption of transitivity requires that those
pairwise comparisons be linked together in a consistent way.

\section{Exercises}

\chex{Let $X= \mathbb{R}^2_+$. Verify that $X$ satisfies all five properties required of a
consumption set in Assumption 1.1.}

\chex*{\label{chex:1-2}Let $\gtrsim$ be a preference relation. Prove the following:
\begin{enumerate}[\quad(1)] 
\item $\succsim \in \succsim$
\item $\sim \in \succsim$
\item $\succ \cup \sim = \succsim$
\item $\succ \cap \sim = \varnothing$
\end{enumerate}
}%
{Use the definition.}

\chex{Give a proof or convincing argument for each of the following claims made in the
text. \begin{enumerate}[\quad(1)] \item Neither $\succ$ nor $\sim$ is complete. \item For
any $x_1$ and $x_2$ in $X$, only one of the following holds: $x_1 \succ x_2$, or $x_2
\succ x_1$, or $x_1 \sim x_2$. \end{enumerate} }


\chex*{\label{chex:1-4}Prove that if $\succsim$ is a preference relation, then the
relation $\succ$ is transitive and the relation $\sim$ is transitive. Also show that if
$x_1 \sim x_2 \succsim x_3$, then $x_1 \succsim x_3$.}% 
{To get you started, take the indifference relation. Consider any three points $x_i \in
X$, $i=1, 2, 3$, where $x_1 \sim x_2$ and $x_2 \sim x_3$. We want to show that $x_1 \sim
x_2$ and $x_2 \sim x_3$ $\Rightarrow x_1 \sim x_3$. By definition of $\sim$, $x_1 \sim x_2
\Rightarrow x_1 \succsim x_2$ and $x_2 \succsim x_1$. Similarly, $x_2 \sim x_3 \Rightarrow
x_2 \succsim x_3$. By transivity of $\succsim$, $x_1 \succsim x_2$ and $x_2 \succsim x_3
\Rightarrow x_1 \succsim x_3$. Keep going.}

\chex{If $\succsim$ is a preference relation, prove the following: For any $x_0 \in X$,
\begin{enumerate}[\quad(1)]
\item $\sim (x_0) = \succsim (x_0) \cap \succsim (x_0)$
\item $ \succsim(x_0) = \sim (x_0) \cup \succ (x_0)$
\item $\sim (x_0) \cap \succ (x_0) = \varnothing$
\item $ \sim (x_0) \cap \succ (x_0) = \varnothing$
\item $ \succ (x_0) \cap \succ(x_0) = \varnothing$
\item $\succ (x_0) \cap \sim (x_0) \cap \succ (x_0) = \varnothing$
\item $\succ (x_0) \cup \sim (x_0) \cup \succ (x_0) = X$
\end{enumerate}
}

\chex{Cite a credible example where the preferences of an ‘ordinary consumer’ would be unlikely to satisfy
the axiom of convexity.}

\chex{Prove that under Axiom 5a, the set $\succsim (x_0)$ is a convex set for any $x_0 \in X$.}

\chex{Sketch a map of indifference sets that are all parallel, negatively sloped straight
lines, with preference increasing north-easterly. We know that preferences such as these
satisfy Axioms 1, 2, 3, and 4. Prove that they also satisfy Axiom 5a. Prove that they do
not satisfy Axiom 5.}

\chex{Sketch a map of indifference sets that are all parallel right angles that ‘kink’ on
the line $x_1 = x_2$. If preference increases north-easterly, these preferences will
satisfy Axioms 1, 2, 3, and 4a. Prove that they also satisfy Axiom 5a. Do they satisfy
Axiom 4? Do they satisfy Axiom 5?}

\chex*{\label{chex:1-10}Suppose the $\succsim$ is a binary relation over gambles in
$\mathcal{G}$ satisfying Axioms G2, G3, and G4. Show that $\succsim$ satisfies G1 as
well.}%
{Use a method similar to that employed in (1.11) to eliminate the Lagrangian multiplier
and reduce $(n+1)$ conditions to only $n$ conditions.}

\chex{Show that if $\succsim$ is continuous, then the sets $A$ and $B$ defined in the
proof of Theorem 1.1 are closed subsets of $\mathbb{R}$.}

\backmatter

%%% hints and answers
\printHintsAnswers

%%% terms
\printTerm

\end{document}
