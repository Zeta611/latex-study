\documentclass[a4paper,amsmath,itemph]{oblivoir}

\usepackage{fapapersize}
\usefapapersize{*,*,1in,*,1in,*}

\makeatletter
\let\ATonum\@onum
\makeatother

\usepackage{relsize}
\usepackage{tcolorbox}
\tcbuselibrary{listings,breakable}
\tcbset{listing engine=listings,colframe=cyan,colback=cyan!5!white}

\setmainfont{Noto Serif}
\setsansfont{Noto Sans}
\setmonofont{Noto Sans Mono}
\setkomainfont[Noto Serif CJK KR]()(Bold)(Medium)
\setkosansfont[Noto Sans CJK KR]()(Bold)(Medium)
\usepackage{unicode-math}
\setmathfont{libertinusmath-regular.otf}

\setlength{\parindent}{0mm}

\ExplSyntaxOn

\NewDocumentEnvironment {intro} {o}
{
  \IfValueTF { #1 }
  {
    \int_set:Nn \l_tmpa_int { #1 }
  }
  {
    \int_set:Nn \l_tmpa_int { 1 }
  }
  \noindent \rule {\linewidth}{3pt}
  \par 
  \sffamily [No.\space\int_use:N \l_tmpa_int ]\ \ 
  \bfseries
}
{
  \hfill 2019년~6월~27일
  \par
  \vskip -.3\baselineskip
  \noindent \rule {\linewidth}{1pt}
  \par
  \vskip .5\baselineskip
}

\cs_new:Npn \hello_fn_one:n #1
  {
    Hello, ~ \textit{#1} !
  }

\NewDocumentCommand \hellocmdone { m }
  {
    \hello_fn_one:n { #1 }
  }

\cs_new:Npn \hello_fn_two:n #1
  {
    Hello, ~ \tl_mixed_case:n { #1 } !
  }

\NewDocumentCommand \hellocmdtwo { m }
  {
    \hello_fn_two:n { #1 }
  }
\ExplSyntaxOff

\begin{document}

\begin{intro}
  Hello, world
\end{intro}

이재호 풀이.

\medskip

\begin{tcolorbox}[title={연습문제 1},fonttitle=\sffamily]
문제에서 제시한 \verb|\hellocmd|에서 인자로 주어진 이름을 이탤릭체로 식자하도록 해보아라.

\tcblower

\begin{verbatim}
\ExplSyntaxOn
\cs_new:Npn \hello_fn:n #1
  {
    Hello, ~ \textit{#1} !
  }

\NewDocumentCommand \hellocmd { m }
  {
    \hello_fn:n { #1 }
  }
\ExplSyntaxOff
\end{verbatim}

\begin{tcolorbox}{}
입력: \verb|\hellocmd{expl3}|\\
출력: \hellocmdone{expl3}
\end{tcolorbox}

\end{tcolorbox}

\begin{tcolorbox}[title={연습문제 2},fonttitle=\sffamily]
문제에서 제시한 \verb|\hellocmd|를, 넘어온 인자의 첫 글자를 대문자로 바꾸어 출력하도록 작성하여라.

\tcblower

\begin{verbatim}
\ExplSyntaxOn
\cs_new:Npn \hello_fn:n #1
  {
    Hello, ~ \tl_mixed_case:n { #1 } !
  }

\NewDocumentCommand \hellocmd { m }
  {
    \hello_fn:n { #1 }
  }
\ExplSyntaxOff
\end{verbatim}

\begin{tcolorbox}{}
입력: \verb|\hellocmd{expl3}|\\
출력: \hellocmdtwo{expl3}
\end{tcolorbox}

\end{tcolorbox}

\end{document}
