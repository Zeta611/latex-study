\documentclass[a4paper,amsmath,itemph]{oblivoir}

\usepackage{fapapersize}
\usefapapersize{*,*,1in,*,1in,*}

\makeatletter
\let\ATonum\@onum
\makeatother

\usepackage{relsize}
\usepackage{tcolorbox}
\tcbuselibrary{listings,breakable}
\tcbset{listing engine=listings,colframe=cyan,colback=cyan!5!white}

\setmainfont{Noto Serif}
\setsansfont{Noto Sans}
\setmonofont{Noto Sans Mono}
\setkomainfont[Noto Serif CJK KR]()(Bold)(Medium)
\setkosansfont[Noto Sans CJK KR]()(Bold)(Medium)
\usepackage{unicode-math}
\setmathfont{libertinusmath-regular.otf}

\setlength{\parindent}{0mm}

\ExplSyntaxOn

\NewDocumentEnvironment {intro} {o}
  {
    \IfValueTF { #1 }
      {
        \int_set:Nn \l_tmpa_int { #1 }
      }
      {
        \int_set:Nn \l_tmpa_int { 1 }
      }
    \noindent \rule {\linewidth}{3pt}
    \par 
    \sffamily [No.\space\int_use:N \l_tmpa_int ]\ \ 
    \bfseries
  }
{
  \hfill 2019년~6월~29일
  \par
  \vskip -.3\baselineskip
  \noindent \rule {\linewidth}{1pt}
  \par
  \vskip .5\baselineskip
}

\ExplSyntaxOn
\cs_new:Npn \foo_fn:o #1
  {
    \tl_set:Nx \l_tmpa_tl { #1 }
    \tl_if_empty:oF { \l_tmpa_tl }
      {
        \fbox { \tl_head:N \l_tmpa_tl }
        \foo_fn:o { \tl_tail:N \l_tmpa_tl }
      }
  }

\NewDocumentCommand \foobox { m }
  {
    \foo_fn:o { #1 }
  }
\ExplSyntaxOff

\begin{document}

\begin{intro}[2]
  Ready, Set, Go
\end{intro}

이재호 풀이.

\medskip

\begin{tcolorbox}[title={문제 1},fonttitle=\sffamily\bfseries]
  문자열 인자를 받아서 각 글자마다 \verb|\fbox|를 치는 명령 \verb|\foobox|를
  작성하여라.

  \tcblower

  \begin{verbatim}
\ExplSyntaxOn
\cs_new:Npn \foo_fn:o #1
  {
    \tl_set:Nx \l_tmpa_tl { #1 }
    \tl_if_empty:oF { \l_tmpa_tl }
      {
        \fbox { \tl_head:N \l_tmpa_tl }
        \foo_fn:o { \tl_tail:N \l_tmpa_tl }
      }
  }

\NewDocumentCommand \foobox { m }
  {
    \foo_fn:o { #1 }
  }
\ExplSyntaxOff
  \end{verbatim}

  \begin{tcolorbox}{}
    입력: \verb|\foobox{expl3}|\\
    출력: \foobox{expl3}
  \end{tcolorbox}

\end{tcolorbox}


\begin{tcolorbox}[title={문제 3},fonttitle=\sffamily\bfseries]
  다음 실행의 결과가 어떠할지 예측해보아라.
  실제로 예상과 같은지 확인해보아라.

  \tcblower

  \begin{verbatim}
\ExplSyntaxOn
\tl_new:N \l_tmpc_tl

\tl_set:Nn \l_tmpa_tl { foo }
\tl_set:Nn \l_tmpb_tl { \l_tmpa_tl }
\tl_set:No \l_tmpc_tl { \l_tmpa_tl }

\tl_set:Nn \l_tmpa_tl { bar }

\tl_use:N \l_tmpb_tl
\tl_use:N \l_tmpc_tl
\ExplSyntaxOff
  \end{verbatim}

  \begin{tcolorbox}{}
    출력: barfoo
  \end{tcolorbox}

\end{tcolorbox}

\end{document}
